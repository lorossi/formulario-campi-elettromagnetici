\documentclass{article}
\usepackage[utf8]{inputenc}
\usepackage[italian]{babel}
\usepackage{verbatim}
\usepackage{geometry}
\usepackage{enumitem}
\usepackage{amssymb}
\usepackage{amsmath}
\usepackage{tikz}

\geometry{left = 2 cm, right = 2 cm, bottom = 2 cm, top = 2 cm}
\setlength{\parindent}{0em}

\title{Appunti di Campi Elettromagnetici}
\author{Lorenzo Rossi}
\date{AA 2019/2020}

\begin{document}
\maketitle
\section{Operatori differenziali}

\subsection{Gradiente}
\`E un operatore definito su un \textbf{campo scalare} e da come risultato un \textbf{campo vettoriale} che punta alla direzione di maggior variazione del campo.
\[ \nabla \; \Phi (x, y, z) = E_x(x, y, z) + E_y(x, y, z) + E_z(x, y, z) \]

\subsection{Divergenza}
\`E un operatore definito su un \textbf{campo vettoriale} e da come risultato un \textbf{campo scalare} che si annulla in presenza di sorgenti e:
\begin{itemize}
	\item \textbf{Positiva} in corrispondenza di sorgenti - nel caso di campo elettrostatico, una carica \textbf{positiva}
	\item \textbf{Negativa} in corrispondenza di pozzi - nel caso di campo elettrostatico, una carica \textbf{negativa}
\end{itemize}
\[ \nabla \cdot \Phi (x, y, z) = \frac{\partial F_x(x, y, z)}{\partial x} + \frac{\partial F_y(x, y, z)}{\partial y} + \frac{\partial F_z(x, y, z)}{\partial z} \]

\subsection{Rotore}
\`E un operatore definito su un \textbf{campo vettoriale} in \( \mathbb{R} ^ 3 \) e da come risultato un \textbf{campo vettoriale} che indica la direzione ed il modulo dei vortici nel campo.
\[ \nabla \times \Phi (x, y, z) = det  \begin{bmatrix} \bar{u_x} && \bar{u_y} && \bar{u_z} \\ \frac{\partial }{\partial x} && \frac{\partial }{\partial y} && \frac{\partial }{\partial z} \\ F_x && F_y && F_z  \end{bmatrix} = (\frac{\partial F_z}{\partial y} - \frac{\partial F_y}{\partial z}) \; \bar{u_x} + (\frac{\partial F_x}{\partial z} - \frac{\partial F_z}{\partial x}) \; \bar{u_y} + (\frac{\partial F_y}{\partial x} - \frac{\partial F_x}{\partial y}) \; \bar{u_z} \]


\section{Formule}

\subsection{Condensatore}
\begin{itemize}
	\item \( C = \frac{Q}{V} = \frac{•}{•}
\end{itemize}
\end{document}